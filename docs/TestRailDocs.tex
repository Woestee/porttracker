\documentclass{article}
\usepackage[hidelinks]{hyperref}
\usepackage{array}

\title{TestRail Documentation \\ PortTracker}
\author{CS 3338 -- Final Project Team}
\date{\today}

\begin{document}
\maketitle

\section{Introduction}

This document summarizes the structure and results of the PortTracker
test suite as executed in TestRail. Two test runs were completed to
align with the development snapshots defined in the project: Snapshot 2
(core functionality) and Snapshot 3 (final functionality including API
pricing and delete operations).

PortTracker is a Flask-based stock portfolio tracker using SQLite for
storage and Alpha Vantage for live price retrieval.

\section{TestRail Structure}

\begin{itemize}
    \item \textbf{Project:} PortTracker
    \item \textbf{Test Suite:} Add Case
    \item \textbf{Total Test Cases:} 6
    \item \textbf{Snapshots Tested:}
    \begin{itemize}
        \item Snapshot 2 Test Run
        \item Snapshot 3 Test Run
    \end{itemize}
\end{itemize}

The test cases included in the suite are:

\begin{itemize}
    \item PT-001 Add Valid Stock
    \item PT-002 Add Stock with Empty Fields
    \item PT-003 Add Stock with Invalid Shares
    \item PT-004 Invalid Ticker / API Failure
    \item PT-005 Dashboard Displays Total Portfolio Value
    \item PT-006 Delete Stock from Dashboard
\end{itemize}

\section{Snapshot 2 Test Run}

\subsection*{Included Test Cases}
The following four cases represent the core application features before
external pricing and deletion capabilities were added:

\begin{itemize}
    \item PT-001 Add Valid Stock
    \item PT-002 Add Stock with Empty Fields
    \item PT-003 Add Stock with Invalid Shares
    \item PT-005 Dashboard Displays Total Portfolio Value
\end{itemize}

\subsection*{Results Summary}

All Snapshot 2 test cases passed successfully using placeholder pricing
and early database integration.

\begin{itemize}
    \item \textbf{PT-001: PASS}
    \item \textbf{PT-002: PASS}
    \item \textbf{PT-003: PASS}
    \item \textbf{PT-005: PASS}
\end{itemize}

\subsection*{Exported Test Run}

The exported PDF containing the detailed pass results is included in:

\begin{center}
\texttt{docs/test\_results/Snapshot 2 Test Run - TestRail.pdf}
\end{center}

\section{Snapshot 3 Test Run}

\subsection*{Included Test Cases}

Snapshot 3 includes 2 functional test cases, covering API
integration and deletion functionality:

\begin{itemize}
    \item PT-004 Invalid Ticker / API Failure
    \item PT-006 Delete Stock from Dashboard
\end{itemize}

\subsection*{Results Summary}

Most tests passed successfully. PT-004 may occasionally be marked
\textbf{Blocked} due to Alpha Vantage API rate limits, which is a known
and acceptable constraint for this project.

\begin{itemize}
    \item \textbf{PT-004: PASS or BLOCKED} (depending on API response)
    \item \textbf{PT-006: PASS}
\end{itemize}

\subsection*{Exported Test Run}

The PDF export for Snapshot 3 is located at:

\begin{center}
\texttt{docs/test\_results/Snapshot 3 Test Run - TestRail.pdf}
\end{center}

\section{Conclusion}

The executed test runs demonstrate that PortTracker meets its functional
requirements across both core and extended features. Snapshot 2 validated
basic add/list behaviors, while Snapshot 3 confirmed full integration
with the Alpha Vantage API and the portfolio deletion workflow. All test
results align with project expectations, and the application behaved
correctly under normal and invalid input scenarios.

\end{document}
