\documentclass{article}
\usepackage[hidelinks]{hyperref}
\usepackage{array}

\title{Design Specification \\ PortTracker}
\author{CS 3338 -- Final Project Team}
\date{\today}

\begin{document}
\maketitle

\section{Introduction}

This document describes the user interface and behavioral design for the
PortTracker web application. PortTracker is a Dockerized Flask
application that allows users to track stock holdings using a local
SQLite database and live prices fetched from the Alpha Vantage API.

The purpose of this Design Specification is to provide a clear, user
facing description of each screen, what inputs it accepts, and how the
system responds. It is intended to complement the SRS and SDD documents.

\section{Overview of Screens and Routes}

PortTracker exposes four main routes:

\begin{center}
\begin{tabular}{|>{\raggedright}p{2.5cm}|>{\raggedright}p{3cm}|>{\raggedright}p{7cm}|}
\hline
\textbf{Route} & \textbf{Screen Name} & \textbf{Purpose} \tabularnewline
\hline
\texttt{/} & Home & Landing page with brief description of the app and
navigation links to the Dashboard and Add Stock pages. \tabularnewline
\hline
\texttt{/dashboard} & Dashboard & Shows the list of all stock holdings
in the portfolio, including ticker, shares, price, value, and total
portfolio value. Provides Remove actions for each row. \tabularnewline
\hline
\texttt{/add-stock} & Add Stock & Form where the user can enter a stock
ticker and number of shares to add to the portfolio. On submit, the app
fetches the latest price and stores the holding. \tabularnewline
\hline
\texttt{/delete-stock/<id>} & Delete Stock & Route invoked via POST
when the user clicks the Remove button for a holding on the Dashboard.
Deletes the specified row and redirects back to the Dashboard. \tabularnewline
\hline
\end{tabular}
\end{center}

Each screen shares a common navigation bar defined in \texttt{base.html}
with links to Home, Dashboard, and Add Stock.

\section{Common Layout}

All pages extend the \texttt{base.html} template, which includes:

\begin{itemize}
    \item A top navigation bar with links to:
    \begin{itemize}
        \item Home (\texttt{/})
        \item Dashboard (\texttt{/dashboard})
        \item Add Stock (\texttt{/add-stock})
    \end{itemize}
    \item A horizontal rule separating navigation from page content.
    \item A flash message area that displays success or error messages.
    \item A main content block where each page injects its own content.
\end{itemize}

Styling is provided via \texttt{static/css/style.css}. The design is
simple and readable, with a focus on clearly displaying tabular data.

\section{Home Screen (/)}

\subsection*{Purpose}

Provide a simple entry point to the application and direct the user to
the Dashboard or Add Stock screens.

\subsection*{Layout and Content}

\begin{itemize}
    \item Navigation bar (from \texttt{base.html}).
    \item Page title: \textbf{PortTracker}.
    \item Short description of the application.
    \item Link or button to navigate to the Dashboard.
\end{itemize}

\subsection*{Behavior}

\begin{itemize}
    \item No user input is required on this page.
    \item Clicking the Dashboard link takes the user to \texttt{/dashboard}.
    \item Clicking Add Stock in the nav bar takes the user to \texttt{/add-stock}.
\end{itemize}

\section{Dashboard Screen (/dashboard)}

\subsection*{Purpose}

Display the current portfolio, including all holdings stored in the
SQLite database, and provide the ability to remove holdings.

\subsection*{Layout and Content}

\begin{itemize}
    \item Navigation bar (from \texttt{base.html}).
    \item Header: \textbf{Your Portfolio}.
    \item Paragraph showing total portfolio value:
    \begin{itemize}
        \item Example: \texttt{Total value: \$1234.56}
    \end{itemize}
    \item Table with the following columns:
    \begin{itemize}
        \item Ticker
        \item Shares
        \item Price
        \item Value
        \item Actions (Remove button)
    \end{itemize}
    \item If there are no holdings, display a single row indicating that
    there are no stocks yet and include a link to Add Stock.
\end{itemize}

\subsection*{Data Sources}

\begin{itemize}
    \item The Dashboard calls \texttt{db.get\_portfolio()} to retrieve
    all holdings from the SQLite \texttt{holdings} table.
    \item Each holding includes an internal \texttt{id}, \texttt{ticker},
    \texttt{shares}, \texttt{price}, and computed \texttt{value}.
\end{itemize}

\subsection*{User Interactions and Behavior}

\begin{itemize}
    \item When the user navigates to \texttt{/dashboard}, the page
    always shows the latest state of the database.
    \item Clicking the Remove button submits a POST request to
    \texttt{/delete-stock/<id>} for the corresponding holding.
    \item On successful deletion, the user is redirected back to the
    Dashboard, the row is removed, and a confirmation flash message is
    displayed.
\end{itemize}

\section{Add Stock Screen (/add-stock)}

\subsection*{Purpose}

Allow users to add new stock holdings to the portfolio by entering a
ticker symbol and number of shares.

\subsection*{Layout and Content}

\begin{itemize}
    \item Navigation bar (from \texttt{base.html}).
    \item Header: \textbf{Add Stock}.
    \item Form fields:
    \begin{itemize}
        \item \textbf{Ticker} (text input, required).
        \item \textbf{Shares} (number input, minimum value 1, required).
    \end{itemize}
    \item Submit button labeled \textbf{Add}.
\end{itemize}

\subsection*{Behavior}

\paragraph{GET Request}
\begin{itemize}
    \item On a GET request, the form is displayed with empty fields.
\end{itemize}

\paragraph{POST Request}
When the user submits the form:

\begin{enumerate}
    \item The application reads the \texttt{ticker} and \texttt{shares}
    values from the form.
    \item Input validation is performed:
    \begin{itemize}
        \item Ticker must be non-empty.
        \item Shares must be a positive integer.
    \end{itemize}
    \item If validation fails, a flash message is shown (e.g., “Ticker and
    Shares are required” or “Shares must be a positive integer”) and the
    user is redirected back to the Add Stock page.
    \item If validation succeeds, the application calls
    \texttt{alpha.get\_latest\_price(ticker)} to retrieve the current
    price from Alpha Vantage.
    \item If the price lookup fails, a flash message is shown (e.g.,
    “Could not fetch price for TICKER”) and the user is redirected back
    to the Add Stock page without saving anything.
    \item If the price lookup succeeds, \texttt{db.add\_holding()} is
    called to insert the new holding into the SQLite database.
    \item A success flash message is shown and the user is redirected to
    the Dashboard.
\end{enumerate}

\section{Delete Stock Route (/delete-stock/<id>)}

\subsection*{Purpose}

Handle deletion of an existing holding from the portfolio.

\subsection*{Behavior}

\begin{itemize}
    \item This route is invoked via a POST request from the Remove button
    on the Dashboard.
    \item The \texttt{holding\_id} parameter identifies which row to
    delete.
    \item The route calls \texttt{db.delete\_holding(holding\_id)}.
    \item After deletion, it flashes a confirmation message and redirects
    back to \texttt{/dashboard}.
    \item If the specified ID does not exist, the current implementation
    simply performs the delete and redirects; no additional error is
    shown to the user.
\end{itemize}

\section{Navigation Flow}

At a high level, the navigation flow is:

\begin{enumerate}
    \item User lands on the Home page (\texttt{/}) and reads a brief
    description of PortTracker.
    \item User clicks \textbf{Dashboard} to see current holdings.
    \item User clicks \textbf{Add Stock} to add a new position.
    \item User is redirected back to the Dashboard to see the updated
    table.
    \item User may remove a holding by clicking \textbf{Remove}, which
    posts to \texttt{/delete-stock/<id>} and then returns to the
    Dashboard.
\end{enumerate}

This simple flow supports the main use case for the project: tracking a
set of stock holdings with persistent storage and basic portfolio
management.

\end{document}

