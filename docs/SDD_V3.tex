\documentclass{article}
\usepackage[margin=1in]{geometry}
\usepackage{hyperref}
\usepackage{graphicx}
\usepackage{array}

\title{Software Design Document (SDD) \\
\large PortTracker}
\author{
Dante Fernandez \\
Marco Lopez \\
Bryan Valencia \\
Alexander Woeste}
\date{Snapshot 3}

\begin{document}
\maketitle

\section*{Version Table}
\begin{tabular}{|c|c|p{7cm}|}
\hline
Version & Date & Notes \\
\hline
1.0 & Snapshot 1 & Original plan using React, Node, and PostgreSQL. Nothing was implemented in that version. \\
\hline
2.0 & Snapshot 2 & System changed to Flask and SQLite. Added Add Stock, Dashboard, data storage, and basic flow. No price API yet. \\
\hline
3.0 & Snapshot 3 & Final build. Real time price lookup added, delete feature added, improved workflow, better error handling, and cleaner interface with flash messages. \\
\hline
\end{tabular}

\newpage
\tableofcontents
\newpage

\section{Revision Summary for Version 3.0}

    Snapshot 3 finishes the full PortTracker system. It goes past the basic setup we had in Snapshot 2 and adds the important features that make the app work like an actual portfolio tracker.

The updates for Snapshot 3 include
\begin{itemize}
    \item Real time price fetching using the Alpha Vantage API
    \item Delete Stock route for removing entries from the database
    \item Automatic price lookup when a user adds a new stock
    \item More accurate total value calculations on the Dashboard
    \item Better error handling for invalid tickers and API failures
    \item Updated workflow and data flow across pages
    \item User interface improvements such as flash messages and smoother page transitions
\end{itemize}

This completes the project for Snapshot 3.

\section{Introduction}

\subsection{Purpose}
This document explains how the final version of PortTracker is built and how all the pieces fit together. It covers the architecture, the different modules, the database design, the price API, the templates, and the entire workflow from adding a stock to viewing the portfolio. It is meant for anyone who needs to understand the internal design of the app or plans to build on it later.

\subsection{Intended Audience}
This document is intended for
\begin{itemize}
    \item Students who continue the project in the future
    \item Instructors who want to review the final technical design
    \item Developers who want to add features such as alerts or charts
\end{itemize}

\subsection{Overview}
Snapshot 3 includes
\begin{itemize}
    \item Flask backend application
    \item SQLite database for storing stock data
    \item Alpha Vantage API integration for price lookups
    \item Add Stock, Dashboard, and Delete Stock routes
    \item Docker setup for consistent deployment
\end{itemize}

\section{System Architecture}

\subsection{High Level Architecture}

\begin{verbatim}
[User Browser]
       |
       v
[Flask Application]
       |
       |---- uses API ---->  [Alpha Vantage Server]
       |
       v
[SQLite Database]
\end{verbatim}

\subsection{Design Rationale}
\begin{itemize}
    \item Flask keeps the app simple and easy to maintain
    \item SQLite is perfect for a small local project and needs no setup
    \item Real time price lookups allow the Dashboard to feel more realistic
    \item Docker makes it easier to run the app anywhere without environment issues
\end{itemize}

\section{Module Breakdown}

\subsection{Backend Modules}

\subsubsection{app.py}
This file contains all the main routes of the app.  
Routes include
\begin{itemize}
    \item Slash route for the home page
    \item Dashboard route to show all holdings
    \item Add Stock route with GET and POST behavior
    \item Delete Stock route for removing entries
\end{itemize}

Responsibilities include
\begin{itemize}
    \item Displaying flash messages
    \item Validating form submissions
    \item Looking up prices through the API
    \item Sending and receiving data from the database functions
\end{itemize}

\subsubsection{db.py}
Handles all interactions with the SQLite database.

Functions include
\begin{itemize}
    \item create table function to ensure the holdings table exists
    \item add holding function to insert a new row
    \item get portfolio function to return all rows
    \item delete holding function to remove a row by id
\end{itemize}

\subsubsection{alpha.py}
Handles all price lookups using the external API.  
The main function
\begin{itemize}
    \item get latest price for a ticker symbol and return the price or return nothing if the lookup fails
\end{itemize}

Extra behavior includes
\begin{itemize}
    \item Checking for invalid tickers
    \item Watching for API rate limits
    \item Returning a safe fallback so the app does not crash
\end{itemize}

\section{Frontend Templates}

\subsubsection{base.html}
This template contains the shared structure for all pages.  
It includes
\begin{itemize}
    \item Navigation bar
    \item Display area for flash messages
\end{itemize}

\subsubsection{index.html}
The landing page with a short introduction and navigation links.

\subsubsection{dashboard.html}
This is the main portfolio page and displays
\begin{itemize}
    \item All holdings stored in SQLite
    \item Calculated value for each row
    \item Total value across all holdings
    \item Remove buttons for deleting entries
\end{itemize}

\subsubsection{add stock.html}
This is the form where users add a new holding.  
It includes
\begin{itemize}
    \item Text box for the ticker
    \item Input for number of shares
    \item Submit button
\end{itemize}

Backend behavior includes
\begin{itemize}
    \item Validating the inputs
    \item Using the API to fetch the most recent price
    \item Saving the row to SQLite
\end{itemize}

\section{Database Design}

\subsection{Table Structure}

\begin{tabular}{|c|c|c|}
\hline
Column & Type & Description \\
\hline
id & INTEGER PRIMARY KEY & Unique row id \\
ticker & TEXT & Stock symbol entered by the user \\
shares & INTEGER & Number of shares owned \\
price & REAL & Most recent price returned by the API \\
\hline
\end{tabular}

\subsection{Notes}
\begin{itemize}
    \item The stored price is from when the user added the stock
    \item The Dashboard recalculates value each time the page loads
\end{itemize}

\section{API Integration}

\subsection{Alpha Vantage API}
This API is used to fetch the most recent price for any ticker the user enters.

If the ticker cannot be found
\begin{itemize}
    \item No data is returned
    \item The app shows a flash message
    \item The stock is not added to the portfolio
\end{itemize}

If the rate limit is reached
\begin{itemize}
    \item The function returns nothing
    \item The user sees an error message instead of the app breaking
\end{itemize}

\section{User Interface and Experience}

\subsection{Dashboard}
The Dashboard displays
\begin{itemize}
    \item Ticker symbol
    \item Shares owned
    \item Price used in the calculation
    \item Total value for that row
    \item A remove button that sends a request to delete the row
\end{itemize}

\subsection{Add Stock}
This page includes
\begin{itemize}
    \item Input field for ticker
    \item Input field for shares
    \item Submit button
\end{itemize}

When submitted
\begin{itemize}
    \item Inputs are validated
    \item A price is fetched from the API
    \item The data is saved to SQLite
    \item User is redirected to the Dashboard
\end{itemize}

\section{Workflow Diagram}

\begin{verbatim}
User fills Add Stock form
       |
       v
Flask checks if the data is valid
       |
       v
Flask calls the API for a price
       |
       v
If the API returns nothing
       |
       v
Show flash message and return to Add Stock
       |
       v
If valid, save the entry to the database
       |
       v
Redirect to Dashboard
       |
       v
Dashboard loads holdings and calculates totals
       |
       v
User removes stock
       |
       v
Entry removed from SQLite
Dashboard refreshes with updated values
\end{verbatim}

\section{Error Handling}

\begin{itemize}
    \item Invalid ticker causes a flash message saying the price could not be fetched
    \item API failures show a message instead of stopping the app
    \item Shares must be greater than zero or the form returns an error
    \item Required fields must be filled out
    \item Invalid delete requests do nothing and do not crash the app
\end{itemize}

\section{Security Considerations}

\begin{itemize}
    \item API key is stored safely in an environment file
    \item No user accounts in this version, so no sensitive data is stored
    \item All form inputs are cleaned before being used in SQL statements
\end{itemize}

\section{Glossary}

\begin{itemize}
    \item Holding  
    A stock entry that the user added
    \item Value  
    The amount of shares multiplied by the stored price
\end{itemize}

\section{References}
\begin{itemize}
    \item Flask documentation
    \item Alpha Vantage API documentation
    \item SQLite documentation
\end{itemize}

\end{document}
