\documentclass{article}
\usepackage[hidelinks]{hyperref}

\title{README / User Manual \\ PortTracker}
\author{CS 3338 -- Final Project Team}
\date{\today}

\begin{document}
\maketitle

\section{Overview}

PortTracker is a containerized web application that helps users track their stock holdings.

The application is built with:
\begin{itemize}
    \item \textbf{Backend:} Python Flask
    \item \textbf{Database:} SQLite (local \texttt{portfolio.db} file)
    \item \textbf{Data Source:} Alpha Vantage API for current stock prices
    \item \textbf{Containerization:} Docker (with \texttt{docker-compose})
\end{itemize}

Users can:
\begin{itemize}
    \item Add stock holdings (ticker + number of shares)
    \item Automatically fetch the latest price for that ticker
    \item View a dashboard of all holdings with current value
    \item Remove holdings from the portfolio
\end{itemize}

\section{Run Instructions}

\subsection*{Prerequisites}

\begin{itemize}
    \item Docker Desktop installed and running
    \item Git (or the ability to download the repository as a ZIP)
    \item An Alpha Vantage API key (free tier is sufficient)
\end{itemize}

\subsection*{Clone the Repository}

\begin{enumerate}
    \item Open a terminal or command prompt.
    \item Clone the repository:
\begin{verbatim}
git clone https://github.com/Woestee/porttracker.git
cd porttracker
\end{verbatim}
\end{enumerate}

\subsection*{Configure Environment Variables}

PortTracker reads configuration from a \texttt{.env} file in the project root.

\begin{enumerate}
    \item Create a file named \texttt{.env} in the root of the repository
    (same level as \texttt{Dockerfile} and \texttt{docker-compose.yml}).
    \item Add the following content, replacing the API key with your own:
\begin{verbatim}
FLASK_SECRET_KEY=dev-secret
FLASK_APP=app.py
FLASK_RUN_HOST=0.0.0.0
PORT=5000
ALPHA_VANTAGE_API_KEY=YOUR_ALPHA_VANTAGE_KEY_HERE
\end{verbatim}
\end{enumerate}

\subsection*{Build and Run with Docker Compose}

From the repository root:

\begin{enumerate}
    \item Start the application:
\begin{verbatim}
docker-compose up
\end{verbatim}
    \item After the container starts, open a browser and visit:
\begin{verbatim}
http://localhost:5000
\end{verbatim}
\end{enumerate}

To stop the application, press \texttt{Ctrl + C} in the terminal running
\texttt{docker-compose}, or run:

\begin{verbatim}
docker-compose down
\end{verbatim}

\section{Usage}

The application exposes three main pages:

\begin{itemize}
    \item \textbf{Home ("/")} -- Simple landing page with navigation links.
    \item \textbf{Dashboard ("/dashboard")} -- Shows the current portfolio.
    \item \textbf{Add Stock ("/add-stock")} -- Form for adding new holdings.
\end{itemize}

\subsection*{Adding a Stock}

\begin{enumerate}
    \item Click \textbf{Add Stock} in the navigation bar.
    \item Enter:
    \begin{itemize}
        \item \textbf{Ticker:} stock symbol (e.g., \texttt{AAPL}, \texttt{MSFT}).
        \item \textbf{Shares:} positive integer number of shares.
    \end{itemize}
    \item Submit the form.
    \item The application will:
    \begin{itemize}
        \item Validate the input.
        \item Call the Alpha Vantage API to get the latest price.
        \item Store the ticker, shares, and price in the SQLite database.
    \end{itemize}
\end{enumerate}

If the ticker is invalid or the API call fails, an error message is shown and the holding is not saved.

\subsection*{Viewing the Portfolio Dashboard}

Navigate to \textbf{Dashboard}.

The table displays:
\begin{itemize}
    \item Ticker -- Stock symbol.
    \item Shares -- Number of shares held.
    \item Price -- Last fetched price from Alpha Vantage.
    \item Value -- Computed as \texttt{shares * price}.
    \item Actions -- Buttons to remove holdings.
\end{itemize}

At the top of the page, the total value of the portfolio is shown as the sum of all holding values.

\subsection*{Removing a Stock}

\begin{enumerate}
    \item On the Dashboard, find the row for the holding you want to remove.
    \item Click the \textbf{Remove} button in the Actions column.
    \item The holding is deleted from the database and the Dashboard is refreshed.
\end{enumerate}

\subsection*{Notes and Limitations}

\begin{itemize}
    \item Prices are fetched on add and stored; they are not continuously updated.
    \item The application is for educational purposes only and is not financial advice.
    \item All data is stored locally in \texttt{portfolio.db} inside the container's mounted volume.
\end{itemize}

\section{Jira Link}

Our Jira project for this application is available at:

\begin{center}
\textbf{https://calstatela-cs3338-fall-2025.atlassian.net/jira/software/projects/POR/boards/67}
\end{center}

\end{document}
